\documentclass[10pt,fleqn]{article}
% \usepackage[journal=rsc]{chemstyle}
% \usepackage{mhchem}
\usepackage{amsmath}
\usepackage{amssymb}
\usepackage{amsfonts}
\usepackage{esint}
\usepackage{bbm}
\usepackage{amscd}
\usepackage{picinpar}
\usepackage{graphicx}
\usepackage{tikz}
\usepackage{tikz-3dplot}
\usepackage{indentfirst}
\usepackage{wrapfig}
\usepackage{units}
\usepackage{textcomp}
\usepackage[utf8x]{inputenc}
% \usepackage{feyn}
% \usepackage{feynmp}
\usepackage{xkeyval}
\usepackage{xargs}
\usepackage{verbatim}
\usepackage{pgfplots}
\usepackage{hyperref}
\usetikzlibrary{
  arrows,
  calc,
  decorations.pathmorphing,
  decorations.pathreplacing,
  decorations.markings,
  fadings,
  positioning,
  shapes
}

\DeclareGraphicsRule{*}{mps}{*}{}
\newcommand{\ud}{\mathrm{d}}
\newcommand{\ue}{\mathrm{e}}
\newcommand{\ui}{\mathrm{i}}
\newcommand{\res}{\mathrm{Res}}
\newcommand{\Tr}{\mathrm{Tr}}
\newcommand{\dsum}{\displaystyle\sum}
\newcommand{\dprod}{\displaystyle\prod}
\newcommand{\dlim}{\displaystyle\lim}
\newcommand{\dint}{\displaystyle\int}
\newcommand{\fsno}[1]{{\!\not\!{#1}}}
\newcommand{\eqar}[1]
{
  \begin{align}
    #1
  \end{align}
}
\newcommand{\texp}[2]{\ensuremath{{#1}\times10^{#2}}}
\newcommand{\dexp}[2]{\ensuremath{{#1}\cdot10^{#2}}}
\newcommand{\eval}[2]{{\left.{#1}\right|_{#2}}}
\newcommand{\paren}[1]{{\left({#1}\right)}}
\newcommand{\lparen}[1]{{\left({#1}\right.}}
\newcommand{\rparen}[1]{{\left.{#1}\right)}}
\newcommand{\abs}[1]{{\left|{#1}\right|}}
\newcommand{\sqr}[1]{{\left[{#1}\right]}}
\newcommand{\crly}[1]{{\left\{{#1}\right\}}}
\newcommand{\angl}[1]{{\left\langle{#1}\right\rangle}}
\newcommand{\tpdiff}[4][{}]{{\paren{\frac{\partial^{#1} {#2}}{\partial {#3}{}^{#1}}}_{#4}}}
\newcommand{\tpsdiff}[4][{}]{{\paren{\frac{\partial^{#1}}{\partial {#3}{}^{#1}}{#2}}_{#4}}}
\newcommand{\pdiff}[3][{}]{{\frac{\partial^{#1} {#2}}{\partial {#3}{}^{#1}}}}
\newcommand{\diff}[3][{}]{{\frac{\ud^{#1} {#2}}{\ud {#3}{}^{#1}}}}
\newcommand{\psdiff}[3][{}]{{\frac{\partial^{#1}}{\partial {#3}{}^{#1}} {#2}}}
\newcommand{\sdiff}[3][{}]{{\frac{\ud^{#1}}{\ud {#3}{}^{#1}} {#2}}}
\newcommand{\tpddiff}[4][{}]{{\left(\dfrac{\partial^{#1} {#2}}{\partial {#3}{}^{#1}}\right)_{#4}}}
\newcommand{\tpsddiff}[4][{}]{{\paren{\dfrac{\partial^{#1}}{\partial {#3}{}^{#1}}{#2}}_{#4}}}
\newcommand{\pddiff}[3][{}]{{\dfrac{\partial^{#1} {#2}}{\partial {#3}{}^{#1}}}}
\newcommand{\ddiff}[3][{}]{{\dfrac{\ud^{#1} {#2}}{\ud {#3}{}^{#1}}}}
\newcommand{\psddiff}[3][{}]{{\frac{\partial^{#1}}{\partial{}^{#1} {#3}} {#2}}}
\newcommand{\sddiff}[3][{}]{{\frac{\ud^{#1}}{\ud {#3}{}^{#1}} {#2}}}
\usepackage{fancyhdr}
\usepackage{multirow}
\usepackage{fontenc}
% \usepackage{tipa}
\usepackage{ulem}
\usepackage{color}
\usepackage{cancel}
\newcommand{\hcancel}[2][black]{\setbox0=\hbox{#2}%
  \rlap{\raisebox{.45\ht0}{\textcolor{#1}{\rule{\wd0}{1pt}}}}#2}
\pagestyle{fancy}
\setlength{\headheight}{67pt}
\fancyhead{}
\fancyfoot{}
\fancyfoot[C]{\thepage}
\fancyhead[R]{}
\renewcommand{\footruleskip}{0pt}
\renewcommand{\headrulewidth}{0.4pt}
\renewcommand{\footrulewidth}{0pt}

\newcommand\pgfmathsinandcos[3]{%
  \pgfmathsetmacro#1{sin(#3)}%
  \pgfmathsetmacro#2{cos(#3)}%
}
\newcommand\LongitudePlane[3][current plane]{%
  \pgfmathsinandcos\sinEl\cosEl{#2} % elevation
  \pgfmathsinandcos\sint\cost{#3} % azimuth
  \tikzset{#1/.estyle={cm={\cost,\sint*\sinEl,0,\cosEl,(0,0)}}}
}
\newcommand\LatitudePlane[3][current plane]{%
  \pgfmathsinandcos\sinEl\cosEl{#2} % elevation
  \pgfmathsinandcos\sint\cost{#3} % latitude
  \pgfmathsetmacro\yshift{\cosEl*\sint}
  \tikzset{#1/.estyle={cm={\cost,0,0,\cost*\sinEl,(0,\yshift)}}} %
}
\newcommand\DrawLongitudeCircle[2][1]{
  \LongitudePlane{\angEl}{#2}
  \tikzset{current plane/.prefix style={scale=#1}}
  % angle of "visibility"
  \pgfmathsetmacro\angVis{atan(sin(#2)*cos(\angEl)/sin(\angEl))} %
  \draw[current plane] (\angVis:1) arc (\angVis:\angVis+180:1);
  \draw[current plane,dashed] (\angVis-180:1) arc (\angVis-180:\angVis:1);
}
\newcommand\DrawLatitudeCircleArrow[2][1]{
  \LatitudePlane{\angEl}{#2}
  \tikzset{current plane/.prefix style={scale=#1}}
  \pgfmathsetmacro\sinVis{sin(#2)/cos(#2)*sin(\angEl)/cos(\angEl)}
  % angle of "visibility"
  \pgfmathsetmacro\angVis{asin(min(1,max(\sinVis,-1)))}
  \draw[current plane,decoration={markings, mark=at position 0.6 with {\arrow{<}}},postaction={decorate},line width=.6mm] (\angVis:1) arc (\angVis:-\angVis-180:1);
  \draw[current plane,dashed,line width=.6mm] (180-\angVis:1) arc (180-\angVis:\angVis:1);
}
\newcommand\DrawLatitudeCircle[2][1]{
  \LatitudePlane{\angEl}{#2}
  \tikzset{current plane/.prefix style={scale=#1}}
  \pgfmathsetmacro\sinVis{sin(#2)/cos(#2)*sin(\angEl)/cos(\angEl)}
  % angle of "visibility"
  \pgfmathsetmacro\angVis{asin(min(1,max(\sinVis,-1)))}
  \draw[current plane] (\angVis:1) arc (\angVis:-\angVis-180:1);
  \draw[current plane,dashed] (180-\angVis:1) arc (180-\angVis:\angVis:1);
}
\newcommand\coil[1]{
  {\rh * cos(\t * pi r)}, {\apart * (2 * #1 + \t) + \rv * sin(\t * pi r)}
}
\makeatletter
\define@key{DrawFromCenter}{style}[{->}]{
  \tikzset{DrawFromCenterPlane/.style={#1}}
}
\define@key{DrawFromCenter}{r}[1]{
  \def\@R{#1}
}
\define@key{DrawFromCenter}{center}[(0, 0)]{
  \def\@Center{#1}
}
\define@key{DrawFromCenter}{theta}[0]{
  \def\@Theta{#1}
}
\define@key{DrawFromCenter}{phi}[0]{
  \def\@Phi{#1}
}
\presetkeys{DrawFromCenter}{style, r, center, theta, phi}{}
\newcommand*\DrawFromCenter[1][]{
  \setkeys{DrawFromCenter}{#1}{
    \pgfmathsinandcos\sint\cost{\@Theta}
    \pgfmathsinandcos\sinp\cosp{\@Phi}
    \pgfmathsinandcos\sinA\cosA{\angEl}
    \pgfmathsetmacro\DX{\@R*\cost*\cosp}
    \pgfmathsetmacro\DY{\@R*(\cost*\sinp*\sinA+\sint*\cosA)}
    \draw[DrawFromCenterPlane] \@Center -- ++(\DX, \DY);
  }
}
\newcommand*\DrawFromCenterText[2][]{
  \setkeys{DrawFromCenter}{#1}{
    \pgfmathsinandcos\sint\cost{\@Theta}
    \pgfmathsinandcos\sinp\cosp{\@Phi}
    \pgfmathsinandcos\sinA\cosA{\angEl}
    \pgfmathsetmacro\DX{\@R*\cost*\cosp}
    \pgfmathsetmacro\DY{\@R*(\cost*\sinp*\sinA+\sint*\cosA)}
    \draw[DrawFromCenterPlane] \@Center -- ++(\DX, \DY) node {#2};
  }
}
\makeatother
\tikzstyle{snakearrow} = [decorate, decoration={pre length=0.2cm,
  post length=0.2cm, snake, amplitude=.4mm,
  segment length=2mm},thick, ->]
%% document-wide tikz options and styles
\tikzset{%
  >=latex, % option for nice arrows
  inner sep=0pt,%
  outer sep=2pt,%
  mark coordinate/.style={inner sep=0pt,outer sep=0pt,minimum size=3pt,
    fill=black,circle}%
}
\addtolength{\hoffset}{-1.3cm}
\addtolength{\voffset}{-2cm}
\addtolength{\textwidth}{3cm}
\addtolength{\textheight}{2.5cm}
\renewcommand{\footskip}{10pt}
\setlength{\headwidth}{\textwidth}
\setlength{\headsep}{20pt}
\setlength{\marginparwidth}{0pt}
\parindent=0pt
\title{Phase of polarization components}

\ifpdf
  % Ensure reproducible output
  \pdfinfoomitdate=1
  \pdfsuppressptexinfo=-1
  \pdftrailerid{}
  \hypersetup{
    pdfcreator={},
    pdfproducer={}
  }
\fi

\begin{document}

\maketitle

\section{Introduction}
After selecting a frame, we can always decompose the polarization into the
$\sigma^+$, $\sigma^-$ and $\pi$ components. While the amplitude of this decomposition
is relatively easy to figure out, their phases depends on the selection of basis.
The basis, and therefore the phase, can be arbitrary as long as it's consistent.
This consistency is trivial when only dealing with polarization vectors
(as long as the same basis is used) but becomes more important when the light starts
to interact with other stuff, e.g. atoms. It is therefore important to ensure that
the selection of polarization basis is consistent with the basis
for the angular momentum states of the atoms
(and practically the phase of the CG-coefficient).\\

Although the mathematically correct way to do this is probably to start with
a consistent definition and derive everything from there,
I'd like to derive/check how to get a consistent definition of the phase
based only on definitions/formulas that we use, to see how the definition
can propagate from one part to another.

\section{Derivation}
\subsection{Angular momentum eigenstate phase}
The convention in quantum mechanics is to relate the angular momentum states
using the ladder operators,
\eqar{
  j_{\pm}=j_x\pm\ui j_y
}
and we have
\eqar{
  j_{\pm}|jm\rangle=C_{\pm}|j(m\pm1)\rangle
}
where $C_{\pm}$ are non-negative constant factors.
This uniquely defines the phase of the angular momentum eigenstates.

\subsection{Tensor operator phase}
When calculating the value of a tensor operator, we use Wigner-Eckart theorem
to convert it to CG coefficients and a reduced matrix element.
The derivation for Wigner-Eckart uses the definition/convention
for the spherical basis,
\eqar{
  V^0=&V_z\\
  V^{\pm1}=&\mp\frac{1}{\sqrt{2}}\paren{V_x\pm\ui V_y}
}

\subsection{Polarization phase}
The light polarization can be expressed in the Cartesian and spherical basis as,
\eqar{
  \sum_{i=x,y,z}\!A_i\hat e_i=\sum_{i=-1,0,1}\!A_i\hat e^i
}
Similarly, the inner product between the polarization and an operator
can be expressed as,
\eqar{
  \sum_{i=x,y,z}\!A_i V_i=\sum_{i=-1,0,1}\!A_i V^i
}
Comparing the two, we can see that the transformation between the spherical
and Cartesian polarization basis should be the same as that for the operators, i.e.
{
  \color{red}
  \eqar{
    \hat e^\pi=&\hat e_z\\
    \hat e^{\sigma^\pm}=&\mp\frac{1}{\sqrt{2}}\paren{\hat e_x\pm\ui\hat e_y}
  }
  that satisfies
  \eqar{
    \vec A=&\sum_{i=-1,0,1}\!A_i\hat e^i
  }
}
and the conjugate basis,
{
  \color{blue}
  \eqar{
    \hat e_\pi=&\hat e_z\\
    \hat e_{\sigma^\pm}=&\mp\frac{1}{\sqrt{2}}\paren{\hat e_x\mp\ui\hat e_y}
  }
  that satisfies,
  \eqar{
    \hat e_i\cdot\hat e^j=&\delta_i^j\\
    A_i=&\hat e_i\cdot\vec A
  }
}
Note that since the basis vectors are complex, this is not the same as the $\hat e^i$s.
\section{Properties of polarization amplitude and phase}
\subsection{Amplitude transformation}
From the relation between the basis, we can get the relation between
the (complex) polarization amplitude in the two basis,
{
  \color{red}
  \eqar{
    A_z=&A_\pi\\
    A_x=&-\frac{1}{\sqrt{2}}\paren{A_{\sigma^+}-A_{\sigma^-}}\\
    A_y=&-\frac{\ui}{\sqrt{2}}\paren{A_{\sigma^+}+A_{\sigma^-}}
  }
}
and
{
  \color{red}
  \eqar{
    A_\pi=&A_z\\
    A_{\sigma^\pm}=&\frac{1}{\sqrt{2}}\paren{\ui A_y\mp A_x}
  }
}
\subsection{Condition for linear polarization}
For linear polarization, we need $\vec A\times\vec A^*=0$, or
\eqar{
  A_xA_y^*=&A_x^*A_y\\
  A_yA_z^*=&A_y^*A_z\\
  A_zA_x^*=&A_z^*A_x
}
From $A_xA_y^*=A_x^*A_y$
\eqar{
  -\paren{A_{\sigma^+}-A_{\sigma^-}}\paren{A_{\sigma^+}^*+A_{\sigma^-}^*}=&\paren{A_{\sigma^+}^*-A_{\sigma^-}^*}\paren{A_{\sigma^+}+A_{\sigma^-}}\\
  \abs{A_{\sigma^+}}=&\abs{A_{\sigma^-}}
}
i.e. the $\sigma^\pm$ components must have the same amplitude.\\

From $A_yA_z^*=A_y^*A_z$ and $A_zA_x^*=A_z^*A_x$,
\eqar{
  A_\pi^*\paren{A_{\sigma^+}+A_{\sigma^-}}=&-A_\pi\paren{A_{\sigma^+}^*+A_{\sigma^-}^*}\\
  A_\pi^*\paren{A_{\sigma^+}-A_{\sigma^-}}=&A_\pi\paren{A_{\sigma^+}^*-A_{\sigma^-}^*}\\
  A_\pi^*A_{\sigma^+}=&-A_\pi A_{\sigma^-}^*
}
Let $\phi_\pi$ and $\phi_{\sigma^\pm}$ be the phase for the three polarization components,
i.e. $A_\pi=\abs{A_\pi}\ue^{\ui\phi_\pi}$ and
$A_{\sigma^\pm}=\abs{A_{\sigma^\pm}}\ue^{\ui\phi_{\sigma^\pm}}$,
also given $\abs{A_{\sigma^+}}=\abs{A_{\sigma^-}}=\abs{A_{\sigma}}$
\eqar{
  \abs{A_\pi}\abs{A_{\sigma}}\paren{\ue^{\ui\paren{\phi_{\sigma^+}+\phi_{\sigma^-}-2\phi_\pi}}-1}=&0
}
i.e. either one of $\abs{A_\pi}$ and $\abs{A_{\sigma}}$ is zero,
or $\phi_\sigma-\phi_\pi=n\pi+\dfrac{\pi}2$,
where {\color{red}$\phi_\sigma\equiv\dfrac{\phi_{\sigma^+}+\phi_{\sigma^-}}{2}$}
is the average phase of the $\sigma$ polarizations.
In another word, unless the polarization is purely $z$ or purely in the $x-y$ plane,
linear polarization is achieved when the phase between $\pi$ polarization
and the average $\sigma$ polarizations is off by $\dfrac\pi2$ ($90^\circ$).

\subsection{$\phi_{HV}$}
We can generalize the result above by introducing a phase difference between
horizontal ($x-y$ plane) and vertical ($z$) polarization.
\eqar{
  \begin{split}
    \vec A=&A_\pi\vec e^\pi+A_{\sigma^+}\vec e^{\sigma^+}+A_{\sigma^-}\vec e^{\sigma^-}\\
    =&A_\pi\vec e_z-\frac{1}{\sqrt{2}}\abs{A_{\sigma^+}}\ue^{\ui\phi_{\sigma^+}}\paren{\hat e_x+\ui\hat e_y}+\frac{1}{\sqrt{2}}\abs{A_{\sigma^-}}\ue^{\ui\phi_{\sigma^-}}\paren{\hat e_x-\ui\hat e_y}
  \end{split}
}
Define {\color{red}$A_H\equiv\dfrac{\abs{A_{\sigma^+}}+\abs{A_{\sigma^-}}}{\sqrt2}$, $A_\Delta\equiv\dfrac{\abs{A_{\sigma^+}}-\abs{A_{\sigma^-}}}{\sqrt2}$, $\theta\equiv\dfrac{\phi_{\sigma^+}-\phi_{\sigma^-}}{2}$}
\eqar{
  \begin{split}
    \vec A=&A_\pi\vec e_z-\frac{\ue^{\ui\phi_{\sigma}}}{2}\paren{\paren{A_H+A_\Delta}\ue^{\ui\theta}\paren{\hat e_x+\ui\hat e_y}-\paren{A_H-A_\Delta}\ue^{-\ui\theta}\paren{\hat e_x-\ui\hat e_y}}\\
    =&A_\pi\vec e_z-\frac{\ue^{\ui\phi_{\sigma}}}{2}\paren{
       A_H\paren{\ue^{\ui\theta}\paren{\hat e_x+\ui\hat e_y}
       -\ue^{-\ui\theta}\paren{\hat e_x-\ui\hat e_y}}
       +A_\Delta\paren{\ue^{\ui\theta}\paren{\hat e_x+\ui\hat e_y}
       +\ue^{-\ui\theta}\paren{\hat e_x-\ui\hat e_y}
       }
       }\\
    =&\abs{A_\pi}\ue^{\ui\phi_{\pi}}\vec e_z-\ue^{\ui\phi_{\sigma}}\paren{
       \ui A_H\paren{\sin\theta\hat e_x+\cos\theta\hat e_y}
       +A_\Delta\paren{\cos\theta\hat e_x-\sin\theta\hat e_y}
       }\\
    =&\ue^{\ui\phi_{\pi}}\paren{\color{red}{
       \abs{A_\pi}\vec e_z
       +A_H \ue^{\ui\phi_{HV}}\paren{\sin\theta\hat e_x+\cos\theta\hat e_y}
       +A_\Delta\ue^{\ui\paren{\phi_{HV}+\pi/2}}
       \paren{\cos\theta\hat e_x-\sin\theta\hat e_y}
       }}
  \end{split}
}
where {\color{red}$\phi_{HV}\equiv\phi_{\sigma}-\phi_\pi-\dfrac{\pi}{2}$}.\\

This describes a polarization with an $x-y$ plane major axis along $\sin\theta\hat e_x+\cos\theta\hat e_y$ with amplitude $A_H$ and minor axis along $\cos\theta\hat e_x-\sin\theta\hat e_y$ with amplitude $A_\Delta$. The phase difference between the $x-y$ plane major axis and the $z$ polarization is $\phi_{HV}$ which, again, is $\dfrac\pi2$ ($90^\circ$) off from the phase difference between average $\sigma$ polarization and $\pi$.

\end{document}
