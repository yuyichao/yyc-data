\documentclass[10pt,fleqn]{article}
% \usepackage[journal=rsc]{chemstyle}
% \usepackage{mhchem}
\usepackage{amsmath}
\usepackage{amssymb}
\usepackage{amsfonts}
\usepackage{esint}
\usepackage{bbm}
\usepackage{amscd}
\usepackage{picinpar}
\usepackage{graphicx}
\usepackage{tikz}
\usepackage{tikz-3dplot}
\usepackage{indentfirst}
\usepackage{wrapfig}
\usepackage{units}
\usepackage{textcomp}
\usepackage[utf8x]{inputenc}
% \usepackage{feyn}
% \usepackage{feynmp}
\usepackage{xkeyval}
\usepackage{xargs}
\usepackage{verbatim}
\usepackage{pgfplots}
\usepackage{hyperref}
\usetikzlibrary{
  arrows,
  calc,
  decorations.pathmorphing,
  decorations.pathreplacing,
  decorations.markings,
  fadings,
  positioning,
  shapes
}

\DeclareGraphicsRule{*}{mps}{*}{}
\newcommand{\ud}{\mathrm{d}}
\newcommand{\ue}{\mathrm{e}}
\newcommand{\ui}{\mathrm{i}}
\newcommand{\res}{\mathrm{Res}}
\newcommand{\Tr}{\mathrm{Tr}}
\newcommand{\dsum}{\displaystyle\sum}
\newcommand{\dprod}{\displaystyle\prod}
\newcommand{\dlim}{\displaystyle\lim}
\newcommand{\dint}{\displaystyle\int}
\newcommand{\fsno}[1]{{\!\not\!{#1}}}
\newcommand{\eqar}[1]
{
  \begin{align}
    #1
  \end{align}
}
\newcommand{\texp}[2]{\ensuremath{{#1}\times10^{#2}}}
\newcommand{\dexp}[2]{\ensuremath{{#1}\cdot10^{#2}}}
\newcommand{\eval}[2]{{\left.{#1}\right|_{#2}}}
\newcommand{\paren}[1]{{\left({#1}\right)}}
\newcommand{\lparen}[1]{{\left({#1}\right.}}
\newcommand{\rparen}[1]{{\left.{#1}\right)}}
\newcommand{\abs}[1]{{\left|{#1}\right|}}
\newcommand{\sqr}[1]{{\left[{#1}\right]}}
\newcommand{\crly}[1]{{\left\{{#1}\right\}}}
\newcommand{\angl}[1]{{\left\langle{#1}\right\rangle}}
\newcommand{\tpdiff}[4][{}]{{\paren{\frac{\partial^{#1} {#2}}{\partial {#3}{}^{#1}}}_{#4}}}
\newcommand{\tpsdiff}[4][{}]{{\paren{\frac{\partial^{#1}}{\partial {#3}{}^{#1}}{#2}}_{#4}}}
\newcommand{\pdiff}[3][{}]{{\frac{\partial^{#1} {#2}}{\partial {#3}{}^{#1}}}}
\newcommand{\diff}[3][{}]{{\frac{\ud^{#1} {#2}}{\ud {#3}{}^{#1}}}}
\newcommand{\psdiff}[3][{}]{{\frac{\partial^{#1}}{\partial {#3}{}^{#1}} {#2}}}
\newcommand{\sdiff}[3][{}]{{\frac{\ud^{#1}}{\ud {#3}{}^{#1}} {#2}}}
\newcommand{\tpddiff}[4][{}]{{\left(\dfrac{\partial^{#1} {#2}}{\partial {#3}{}^{#1}}\right)_{#4}}}
\newcommand{\tpsddiff}[4][{}]{{\paren{\dfrac{\partial^{#1}}{\partial {#3}{}^{#1}}{#2}}_{#4}}}
\newcommand{\pddiff}[3][{}]{{\dfrac{\partial^{#1} {#2}}{\partial {#3}{}^{#1}}}}
\newcommand{\ddiff}[3][{}]{{\dfrac{\ud^{#1} {#2}}{\ud {#3}{}^{#1}}}}
\newcommand{\psddiff}[3][{}]{{\frac{\partial^{#1}}{\partial{}^{#1} {#3}} {#2}}}
\newcommand{\sddiff}[3][{}]{{\frac{\ud^{#1}}{\ud {#3}{}^{#1}} {#2}}}
\usepackage{fancyhdr}
\usepackage{multirow}
\usepackage{fontenc}
% \usepackage{tipa}
\usepackage{ulem}
\usepackage{color}
\usepackage{cancel}
\newcommand{\hcancel}[2][black]{\setbox0=\hbox{#2}%
  \rlap{\raisebox{.45\ht0}{\textcolor{#1}{\rule{\wd0}{1pt}}}}#2}
\pagestyle{fancy}
\setlength{\headheight}{67pt}
\fancyhead{}
\fancyfoot{}
\fancyfoot[C]{\thepage}
\fancyhead[R]{}
\renewcommand{\footruleskip}{0pt}
\renewcommand{\headrulewidth}{0.4pt}
\renewcommand{\footrulewidth}{0pt}

\newcommand\pgfmathsinandcos[3]{%
  \pgfmathsetmacro#1{sin(#3)}%
  \pgfmathsetmacro#2{cos(#3)}%
}
\newcommand\LongitudePlane[3][current plane]{%
  \pgfmathsinandcos\sinEl\cosEl{#2} % elevation
  \pgfmathsinandcos\sint\cost{#3} % azimuth
  \tikzset{#1/.estyle={cm={\cost,\sint*\sinEl,0,\cosEl,(0,0)}}}
}
\newcommand\LatitudePlane[3][current plane]{%
  \pgfmathsinandcos\sinEl\cosEl{#2} % elevation
  \pgfmathsinandcos\sint\cost{#3} % latitude
  \pgfmathsetmacro\yshift{\cosEl*\sint}
  \tikzset{#1/.estyle={cm={\cost,0,0,\cost*\sinEl,(0,\yshift)}}} %
}
\newcommand\DrawLongitudeCircle[2][1]{
  \LongitudePlane{\angEl}{#2}
  \tikzset{current plane/.prefix style={scale=#1}}
  % angle of "visibility"
  \pgfmathsetmacro\angVis{atan(sin(#2)*cos(\angEl)/sin(\angEl))} %
  \draw[current plane] (\angVis:1) arc (\angVis:\angVis+180:1);
  \draw[current plane,dashed] (\angVis-180:1) arc (\angVis-180:\angVis:1);
}
\newcommand\DrawLatitudeCircleArrow[2][1]{
  \LatitudePlane{\angEl}{#2}
  \tikzset{current plane/.prefix style={scale=#1}}
  \pgfmathsetmacro\sinVis{sin(#2)/cos(#2)*sin(\angEl)/cos(\angEl)}
  % angle of "visibility"
  \pgfmathsetmacro\angVis{asin(min(1,max(\sinVis,-1)))}
  \draw[current plane,decoration={markings, mark=at position 0.6 with {\arrow{<}}},postaction={decorate},line width=.6mm] (\angVis:1) arc (\angVis:-\angVis-180:1);
  \draw[current plane,dashed,line width=.6mm] (180-\angVis:1) arc (180-\angVis:\angVis:1);
}
\newcommand\DrawLatitudeCircle[2][1]{
  \LatitudePlane{\angEl}{#2}
  \tikzset{current plane/.prefix style={scale=#1}}
  \pgfmathsetmacro\sinVis{sin(#2)/cos(#2)*sin(\angEl)/cos(\angEl)}
  % angle of "visibility"
  \pgfmathsetmacro\angVis{asin(min(1,max(\sinVis,-1)))}
  \draw[current plane] (\angVis:1) arc (\angVis:-\angVis-180:1);
  \draw[current plane,dashed] (180-\angVis:1) arc (180-\angVis:\angVis:1);
}
\newcommand\coil[1]{
  {\rh * cos(\t * pi r)}, {\apart * (2 * #1 + \t) + \rv * sin(\t * pi r)}
}
\makeatletter
\define@key{DrawFromCenter}{style}[{->}]{
  \tikzset{DrawFromCenterPlane/.style={#1}}
}
\define@key{DrawFromCenter}{r}[1]{
  \def\@R{#1}
}
\define@key{DrawFromCenter}{center}[(0, 0)]{
  \def\@Center{#1}
}
\define@key{DrawFromCenter}{theta}[0]{
  \def\@Theta{#1}
}
\define@key{DrawFromCenter}{phi}[0]{
  \def\@Phi{#1}
}
\presetkeys{DrawFromCenter}{style, r, center, theta, phi}{}
\newcommand*\DrawFromCenter[1][]{
  \setkeys{DrawFromCenter}{#1}{
    \pgfmathsinandcos\sint\cost{\@Theta}
    \pgfmathsinandcos\sinp\cosp{\@Phi}
    \pgfmathsinandcos\sinA\cosA{\angEl}
    \pgfmathsetmacro\DX{\@R*\cost*\cosp}
    \pgfmathsetmacro\DY{\@R*(\cost*\sinp*\sinA+\sint*\cosA)}
    \draw[DrawFromCenterPlane] \@Center -- ++(\DX, \DY);
  }
}
\newcommand*\DrawFromCenterText[2][]{
  \setkeys{DrawFromCenter}{#1}{
    \pgfmathsinandcos\sint\cost{\@Theta}
    \pgfmathsinandcos\sinp\cosp{\@Phi}
    \pgfmathsinandcos\sinA\cosA{\angEl}
    \pgfmathsetmacro\DX{\@R*\cost*\cosp}
    \pgfmathsetmacro\DY{\@R*(\cost*\sinp*\sinA+\sint*\cosA)}
    \draw[DrawFromCenterPlane] \@Center -- ++(\DX, \DY) node {#2};
  }
}
\makeatother
\tikzstyle{snakearrow} = [decorate, decoration={pre length=0.2cm,
  post length=0.2cm, snake, amplitude=.4mm,
  segment length=2mm},thick, ->]
%% document-wide tikz options and styles
\tikzset{%
  >=latex, % option for nice arrows
  inner sep=0pt,%
  outer sep=2pt,%
  mark coordinate/.style={inner sep=0pt,outer sep=0pt,minimum size=3pt,
    fill=black,circle}%
}
\addtolength{\hoffset}{-1.3cm}
\addtolength{\voffset}{-2cm}
\addtolength{\textwidth}{3cm}
\addtolength{\textheight}{2.5cm}
\renewcommand{\footskip}{10pt}
\setlength{\headwidth}{\textwidth}
\setlength{\headsep}{20pt}
\setlength{\marginparwidth}{0pt}
\parindent=0pt
\title{Raman between $F=1/2$ states with dipole transition}

\ifpdf
  % Ensure reproducible output
  \pdfinfoomitdate=1
  \pdfsuppressptexinfo=-1
  \pdftrailerid{}
  \hypersetup{
    pdfcreator={},
    pdfproducer={}
  }
\fi

\begin{document}

\maketitle

\section{Introduction}
As a special case of the general effective Hamiltonian caused by second order
perturbation from dipole transition,
if the ground states have total angular momentum $\dfrac12$,
the only allowed terms in the effective Hamiltonian would be scalar and vector shifts
due the the limited degrees of freedoms.\footnote{This applies even if the transition
  inducing the shift is not a dipole transition but we'll limit the calculation here
  to that from dipole transitions.}

This is a special case that is particularly easy to show and we'll calculate
the effective Hamiltonian of such a system in this note.

\section{Calculation}
We can calculate the effective Hamiltonian in any reference frame of choice
so we can pick a reference frame where the decomposition of the polarization
is the easist to handle. We'll pick the polarization plane\footnote{which must exist
  for monotonic light with pure polarization} as the $x$-$y$ plane
(\textit{for a single beam this means the $z$ axis
  is along the $k$ vector of the beam}),
which means that the polarization will only have $\sigma^\pm$ components.\footnote{
  This choice is also motivated by the general knowledge that the effective $B$
  field from the vector shift would be along the $z$ axis given this choice.
}

\begin{figure}[h]
  \centering
  \begin{tikzpicture}
    \node[left] at (-2.9, 4.8) {$m_F$};

    \node at (-2.1, 4.8) {$-\dfrac32$};
    \node at (-0.7, 4.8) {$-\dfrac12$};
    \node at (0.7, 4.8) {$\dfrac12$};
    \node at (2.1, 4.8) {$\dfrac32$};

    \draw[line width=1,double] (-2.6, 4) -- (-1.6, 4);
    \draw[line width=1] (-1.2, 4) -- (-0.2, 4);
    \draw[line width=1] (0.2, 4) -- (1.2, 4);
    \draw[line width=1,double] (1.6, 4) -- (2.6, 4);

    \draw[->,>=stealth,line width=1,red] (-0.7, 0) -- node[pos=0.7,right] {$\abs{M_1}$} (0.7, 3.6);
    \draw[->,>=stealth,line width=1,red] (0.7, 0) -- node[pos=0.7,right] {$\abs{M_2}$} (2.1, 3.6);
    \draw[->,>=stealth,line width=1,blue] (-0.7, 0) -- node[pos=0.7,left] {$\abs{M_2}$} (-2.1, 3.6);
    \draw[->,>=stealth,line width=1,blue] (0.7, 0) -- node[pos=0.7,left] {$\abs{M_1}$} (-0.7, 3.6);

    \draw[dashed] (-2.6, 3.6) -- (2.9, 3.6);
    \draw[->,>=stealth,line width=0.8] (2.8, 4.3) -- (2.8, 4);
    \draw[->,>=stealth,line width=0.8] (2.8, 3.3) -- (2.8, 3.6);
    \node[right] at (2.8, 3.8) {$\Delta$};

    \draw[line width=1] (-1.2, 0) -- (-0.2, 0);
    \draw[line width=1] (0.2, 0) -- (1.2, 0);
  \end{tikzpicture}
  \caption{Coupling to the excited state with $\sigma^\pm$ polarizations
    and single-photon detuning $\Delta$.
    The double-lined excited states only exists if the exited states have $F=3/2$.
    The allowed coupling between the ground and excited states are drawn in
    {\color{red} red for $\sigma^+$} and {\color{blue} blue for $\sigma^-$}.
    The absolute value of the matrix elements are labeled next to the arrow.
    Note that there are only two distinct values $\abs{M_1}$ and $\abs{M_2}$
    from symmetry. For excited state with $F=1/2$, we would have $M_2=0$.}
  \label{fig:sigma-couple}
\end{figure}

The energy diagram of the system is shown in Fig.~\ref{fig:sigma-couple},
where we can clearly see that the only effect from the light is causing light shift
on the states, as the two ground states never couple to the same excited state
in this frame. Therefore, we just need to calculate the light shift
on the two ground states. The energy shift for the two ground states are,
\eqar{
  E_{1/2}=&\frac{\abs{\Omega_{1/2\rightarrow3/2}}^2+\abs{\Omega_{1/2\rightarrow-1/2}}^2}{4\Delta}\\
  E_{-1/2}=&\frac{\abs{\Omega_{-1/2\rightarrow-3/2}}^2+\abs{\Omega_{-1/2\rightarrow1/2}}^2}{4\Delta}
}
Ignoring the global energy shift (i.e. scalar shift),
the effective Hamiltonian on the ground state is proportional to $\sigma_z$,
and more specifically,
\eqar{
  H_{\mathrm{eff}}=&\frac{E_{1/2}-E_{-1/2}}{2}\sigma_z\label{eq:eff-origin}
}
To obtain the effective Hamiltonian in a different frame,
one can simply replace the $\sigma_z$ matrix with a Pauli matrix
along the appropriate axis in the desired frame.\\

We can further simplify the expression for the effective
Hamiltonian~(eq.~\ref{eq:eff-origin}) to show the dependency on the polarization
and power more directly. Using the matrix elements labeled in
fig.~\ref{fig:sigma-couple}, and assuming the $\sigma^\pm$ polarization have amplitude
$A_\pm$, we can rewrite the effective Hamiltonian
\eqar{
  \begin{split}
    H_{\mathrm{eff}}=&\frac{\sigma_z}{2}\paren{E_{1/2}-E_{-1/2}}\\
    =&\frac{\sigma_z}{8\Delta}\paren{\abs{\Omega_{1/2\rightarrow3/2}}^2+\abs{\Omega_{1/2\rightarrow-1/2}}^2-\abs{\Omega_{-1/2\rightarrow-3/2}}^2-\abs{\Omega_{-1/2\rightarrow1/2}}^2}\\
    =&\frac{\sigma_z}{8\Delta}\paren{\abs{M_2}^2\abs{A_+}^2+\abs{M_1}^2\abs{A_-}^2-\abs{M_1}^2\abs{A_+}^2-\abs{M_2}^2\abs{A_-}^2}\\
    =&\frac{\sigma_z}{8\Delta}\paren{\abs{M_2}^2-\abs{M_1}^2}\paren{\abs{A_+}^2-\abs{A_-}^2}\\
  \end{split}
}
Note that the effective Hamiltonian is proportional to the difference in the amplitude
of $\sigma^+$ and $\sigma^-$ polarization. The effect is $0$ for linear polarization
where $\abs{A_+}=\abs{A_-}$ and is maximized (at a given total power,
i.e. a fixed $\abs{A_+}^2+\abs{A_-}^2$) for circular polarization where
either $A_+=0$ or $A_-=0$.

\end{document}
