\documentclass{beamer}
\mode<presentation>{
  \usetheme{Boadilla}
  \usefonttheme[onlylarge]{structurebold}
  \usefonttheme[stillsansseriflarge]{serif}
  \setbeamerfont*{frametitle}{size=\normalsize,series=\bfseries}
  % \setbeamertemplate{navigation symbols}{}
  \setbeamercovered{transparent}
}
\usepackage[english]{babel}
\usepackage[latin1]{inputenc}
\usepackage{times}
\usepackage[T1]{fontenc}
\usepackage{amsmath}
\usepackage{amssymb}
\usepackage{esint}
\usepackage{hyperref}
\usepackage{tikz}
\usepackage{xkeyval}
\usepackage{xargs}
\usepackage{xcolor}
\usepackage{verbatim}
\usepackage{listings}
\usepackage{multimedia}
\usepackage{bm}
\usepackage{siunitx}
\usetikzlibrary{
  arrows,
  calc,
  decorations.pathmorphing,
  decorations.pathreplacing,
  decorations.markings,
  fadings,
  positioning,
  shapes,
  arrows.meta
}
\tikzset{
  mid arrow/.style={postaction={decorate,decoration={
        markings,
        mark=at position .5 with {\arrow[#1]{stealth}}
      }}},
  mid arrow2/.style={postaction={decorate,decoration={
        markings,
        mark=at position .5 with {\arrow[>=stealth]{><}}
      }}},
}

\usepgfmodule{oo}

\pgfdeclareradialshading{glow2}{\pgfpoint{0cm}{0cm}}{
  color(0mm)=(white);
  color(2mm)=(white);
  color(8mm)=(black);
  color(10mm)=(black)
}
\pgfdeclareradialshading{glow}{\pgfpoint{0cm}{0cm}}{
  color(0mm)=(white);
  color(5mm)=(white);
  color(9mm)=(black);
  color(10mm)=(black)
}

\begin{tikzfadingfrompicture}[name=glow fading]
  \shade [shading=glow] (0,0) circle (1);
\end{tikzfadingfrompicture}

\begin{tikzfadingfrompicture}[name=glow2 fading]
  \shade [shading=glow2] (0,0) circle (1);
\end{tikzfadingfrompicture}

\mode<handout>{
  \usepackage{pgfpages}
  \pgfpagesuselayout{4 on 1}[a4paper,landscape,border shrink=5mm]
  \setbeamercolor{background canvas}{bg=black!10}
}

\newcommand\pgfmathsinandcos[3]{%
  \pgfmathsetmacro#1{sin(#3)}%
  \pgfmathsetmacro#2{cos(#3)}%
}
\newcommand\LongitudePlane[3][current plane]{%
  \pgfmathsinandcos\sinEl\cosEl{#2} % elevation
  \pgfmathsinandcos\sint\cost{#3} % azimuth
  \tikzset{#1/.estyle={cm={\cost,\sint*\sinEl,0,\cosEl,(0,0)}}}
}
\newcommand\LatitudePlane[3][current plane]{%
  \pgfmathsinandcos\sinEl\cosEl{#2} % elevation
  \pgfmathsinandcos\sint\cost{#3} % latitude
  \pgfmathsetmacro\yshift{\cosEl*\sint}
  \tikzset{#1/.estyle={cm={\cost,0,0,\cost*\sinEl,(0,\yshift)}}} %
}
\newcommand\DrawLongitudeCircle[2][1]{
  \LongitudePlane{\angEl}{#2}
  \tikzset{current plane/.prefix style={scale=#1}}
  % angle of "visibility"
  \pgfmathsetmacro\angVis{atan(sin(#2)*cos(\angEl)/sin(\angEl))} %
  \draw[current plane] (\angVis:1) arc (\angVis:\angVis+180:1);
  \draw[current plane,dashed] (\angVis-180:1) arc (\angVis-180:\angVis:1);
}
\newcommand\DrawLatitudeCircleArrow[2][1]{
  \LatitudePlane{\angEl}{#2}
  \tikzset{current plane/.prefix style={scale=#1}}
  \pgfmathsetmacro\sinVis{sin(#2)/cos(#2)*sin(\angEl)/cos(\angEl)}
  % angle of "visibility"
  \pgfmathsetmacro\angVis{asin(min(1,max(\sinVis,-1)))}
  \draw[current plane,decoration={markings, mark=at position 0.6 with {\arrow{<}}},postaction={decorate},line width=.6mm] (\angVis:1) arc (\angVis:-\angVis-180:1);
  \draw[current plane,dashed,line width=.6mm] (180-\angVis:1) arc (180-\angVis:\angVis:1);
}
\newcommand\DrawLatitudeCircle[2][1]{
  \LatitudePlane{\angEl}{#2}
  \tikzset{current plane/.prefix style={scale=#1}}
  \pgfmathsetmacro\sinVis{sin(#2)/cos(#2)*sin(\angEl)/cos(\angEl)}
  % angle of "visibility"
  \pgfmathsetmacro\angVis{asin(min(1,max(\sinVis,-1)))}
  \draw[current plane] (\angVis:1) arc (\angVis:-\angVis-180:1);
  \draw[current plane,dashed] (180-\angVis:1) arc (180-\angVis:\angVis:1);
}
\newcommand\coil[1]{
  {\rh * cos(\t * pi r)}, {\apart * (2 * #1 + \t) + \rv * sin(\t * pi r)}
}
\makeatletter
\define@key{DrawFromCenter}{style}[{->}]{
  \tikzset{DrawFromCenterPlane/.style={#1}}
}
\define@key{DrawFromCenter}{r}[1]{
  \def\@R{#1}
}
\define@key{DrawFromCenter}{center}[(0, 0)]{
  \def\@Center{#1}
}
\define@key{DrawFromCenter}{theta}[0]{
  \def\@Theta{#1}
}
\define@key{DrawFromCenter}{phi}[0]{
  \def\@Phi{#1}
}
\presetkeys{DrawFromCenter}{style, r, center, theta, phi}{}
\newcommand*\DrawFromCenter[1][]{
  \setkeys{DrawFromCenter}{#1}{
    \pgfmathsinandcos\sint\cost{\@Theta}
    \pgfmathsinandcos\sinp\cosp{\@Phi}
    \pgfmathsinandcos\sinA\cosA{\angEl}
    \pgfmathsetmacro\DX{\@R*\cost*\cosp}
    \pgfmathsetmacro\DY{\@R*(\cost*\sinp*\sinA+\sint*\cosA)}
    \draw[DrawFromCenterPlane] \@Center -- ++(\DX, \DY);
  }
}
\newcommand*\DrawFromCenterText[2][]{
  \setkeys{DrawFromCenter}{#1}{
    \pgfmathsinandcos\sint\cost{\@Theta}
    \pgfmathsinandcos\sinp\cosp{\@Phi}
    \pgfmathsinandcos\sinA\cosA{\angEl}
    \pgfmathsetmacro\DX{\@R*\cost*\cosp}
    \pgfmathsetmacro\DY{\@R*(\cost*\sinp*\sinA+\sint*\cosA)}
    \draw[DrawFromCenterPlane] \@Center -- ++(\DX, \DY) node {#2};
  }
}
\makeatother

\newcommand\drawlens[3]{
  % 1: center (x, y)
  % 2: size
  % 3: angle
  \begin{scope}[shift={#1}]
    \node[rotate={#3}] at (0, 0) {\scalebox{#2}{\includegraphics[width=2cm]{fadings/lens.png}}};
  \end{scope}
}
\newcommand\drawwaveplate[3]{
  % 1: center (x, y)
  % 2: size
  % 3: angle
  \begin{scope}[shift={#1}]
    \node[rotate={#3}] at (0, 0)
    {\scalebox{#2}{\includegraphics[width=2cm]{fadings/waveplate.png}}};
  \end{scope}
}
\newcommand\drawaom[4]{
  % 1: center (x, y)
  % 2: xsize
  % 3: ysize
  % 4: angle
  \begin{scope}[shift={#1}]
    \node[rotate={#4}] at (0, 0)
    {\scalebox{#2}[#3]{\includegraphics[width=2cm,height=2cm]{fadings/aom.png}}};
  \end{scope}
  \begin{scope}[rotate around={#4:#1}]
    \fill[orange, even odd rule, opacity=0.8]
    ($#1 + ({#2}, 0)$) arc (0:360:{#2} and {#3})
    -- ($#1 + ({#2}, {#3})$) -- ($#1 + (-{#2}, {#3})$) -- ($#1 + (-{#2}, -{#3})$)
    -- ($#1 + ({#2}, -{#3})$) --cycle;
    \draw ($#1 + ({#2}, {#3})$) -- ($#1 + (-{#2}, {#3})$) -- ($#1 + (-{#2}, -{#3})$)
    -- ($#1 + ({#2}, -{#3})$) --cycle;
  \end{scope}
}
\newcommand\drawpbs[3]{
  % 1: center (x, y)
  % 2: size
  % 3: angle
  \begin{scope}
    \begin{scope}
      \clip[rotate around={#3:#1}] ($#1 - ({#2}, {#2})$) rectangle ($#1 + ({#2}, {#2})$);
      \begin{scope}[transform canvas={shift={#1}, rotate=#3}]
        \node[rotate=-90] at (0, 0)
        {\scalebox{#2}{\includegraphics[width=2cm,
            height=2cm]{fadings/pbs.png}}};
        \draw[line width=1] (-{#2}, -{#2}) -- ({#2}, {#2});
      \end{scope}
    \end{scope}
    % Make sure the frame is not clipped
    \draw[rotate around={#3:#1}] ($#1 - ({#2}, {#2})$) rectangle ($#1 + ({#2}, {#2})$);
  \end{scope}
}
\newcommand\drawnonpbs[3]{
  % 1: center (x, y)
  % 2: size
  % 3: angle
  \begin{scope}
    \begin{scope}
      \clip[rotate around={#3:#1}] ($#1 - ({#2}, {#2})$) rectangle ($#1 + ({#2}, {#2})$);
      \begin{scope}[transform canvas={shift={#1}, rotate=#3}]
        \node[rotate=-90] at (0, 0)
        {\scalebox{#2}{\includegraphics[width=2cm,
            height=2cm]{fadings/non_pbs.png}}};
        \draw[line width=1] (-{#2}, -{#2}) -- ({#2}, {#2});
      \end{scope}
    \end{scope}
    % Make sure the frame is not clipped
    \draw[rotate around={#3:#1}] ($#1 - ({#2}, {#2})$) rectangle ($#1 + ({#2}, {#2})$);
  \end{scope}
}

% not mandatory, but I though it was better to set it blank
\setbeamertemplate{headline}{}
\def\beamer@entrycode{\vspace{-\headheight}}

\tikzstyle{snakearrow} = [decorate, decoration={pre length=0.2cm,
  post length=0.2cm, snake, amplitude=.4mm,
  segment length=2mm},thick, ->]

%% document-wide tikz options and styles

\tikzset{%
  % >=latex, % option for nice arrows
  inner sep=0pt,%
  outer sep=2pt,%
  mark coordinate/.style={inner sep=0pt,outer sep=0pt,minimum size=3pt,
    fill=black,circle}%
}
\tikzset{
  % Define standard arrow tip
  >=stealth',
  % Define style for boxes
  punkt/.style={
    rectangle,
    rounded corners,
    draw=black, very thick,
    text width=8em,
    minimum height=2.5em,
    text centered},
}

\tikzset{onslide/.code args={<#1>#2}{%
    \only<#1>{\pgfkeysalso{#2}}
    % \pgfkeysalso doesn't change the path
  }}
\tikzset{alt/.code args={<#1>#2#3}{%
    \alt<#1>{\pgfkeysalso{#2}}{\pgfkeysalso{#3}}
    % \pgfkeysalso doesn't change the path
  }}
\tikzset{temporal/.code args={<#1>#2#3#4}{%
    \temporal<#1>{\pgfkeysalso{#2}}{\pgfkeysalso{#3}}{\pgfkeysalso{#4}}
    % \pgfkeysalso doesn't change the path
  }}
\newcommand{\ud}{\mathrm{d}}
\newcommand{\ue}{\mathrm{e}}
\newcommand{\ui}{\mathrm{i}}
\newcommand{\res}{\mathrm{Res}}
\newcommand{\Tr}{\mathrm{Tr}}
\newcommand{\dsum}{\displaystyle\sum}
\newcommand{\dprod}{\displaystyle\prod}
\newcommand{\dlim}{\displaystyle\lim}
\newcommand{\dint}{\displaystyle\int}
\newcommand{\fsno}[1]{{\!\not\!{#1}}}
\newcommand{\eqar}[1]
{
  \begin{align*}
    #1
  \end{align*}
}
\newcommand{\texp}[2]{\ensuremath{{#1}\times10^{#2}}}
\newcommand{\dexp}[2]{\ensuremath{{#1}\cdot10^{#2}}}
\newcommand{\eval}[2]{{\left.{#1}\right|_{#2}}}
\newcommand{\paren}[1]{{\left({#1}\right)}}
\newcommand{\lparen}[1]{{\left({#1}\right.}}
\newcommand{\rparen}[1]{{\left.{#1}\right)}}
\newcommand{\abs}[1]{{\left|{#1}\right|}}
\newcommand{\sqr}[1]{{\left[{#1}\right]}}
\newcommand{\crly}[1]{{\left\{{#1}\right\}}}
\newcommand{\angl}[1]{{\left\langle{#1}\right\rangle}}
\newcommand{\tpdiff}[4][{}]{{\paren{\frac{\partial^{#1} {#2}}{\partial {#3}{}^{#1}}}_{#4}}}
\newcommand{\tpsdiff}[4][{}]{{\paren{\frac{\partial^{#1}}{\partial {#3}{}^{#1}}{#2}}_{#4}}}
\newcommand{\pdiff}[3][{}]{{\frac{\partial^{#1} {#2}}{\partial {#3}{}^{#1}}}}
\newcommand{\diff}[3][{}]{{\frac{\ud^{#1} {#2}}{\ud {#3}{}^{#1}}}}
\newcommand{\psdiff}[3][{}]{{\frac{\partial^{#1}}{\partial {#3}{}^{#1}} {#2}}}
\newcommand{\sdiff}[3][{}]{{\frac{\ud^{#1}}{\ud {#3}{}^{#1}} {#2}}}
\newcommand{\tpddiff}[4][{}]{{\left(\dfrac{\partial^{#1} {#2}}{\partial {#3}{}^{#1}}\right)_{#4}}}
\newcommand{\tpsddiff}[4][{}]{{\paren{\dfrac{\partial^{#1}}{\partial {#3}{}^{#1}}{#2}}_{#4}}}
\newcommand{\pddiff}[3][{}]{{\dfrac{\partial^{#1} {#2}}{\partial {#3}{}^{#1}}}}
\newcommand{\ddiff}[3][{}]{{\dfrac{\ud^{#1} {#2}}{\ud {#3}{}^{#1}}}}
\newcommand{\psddiff}[3][{}]{{\frac{\partial^{#1}}{\partial{}^{#1} {#3}} {#2}}}
\newcommand{\sddiff}[3][{}]{{\frac{\ud^{#1}}{\ud {#3}{}^{#1}} {#2}}}

\makeatletter
\newbox\@backgroundblock
\newenvironment{backgroundblock}[2]{%
  \global\setbox\@backgroundblock=\vbox\bgroup%
  \unvbox\@backgroundblock%
  \vbox to0pt\bgroup\vskip#2\hbox to0pt\bgroup\hskip#1\relax%
}{\egroup\egroup\egroup}
\addtobeamertemplate{background}{\box\@backgroundblock}{}
\makeatother

\title{Lamb-Dicke regime/approximation}
\date{}
\author{Yichao Yu}
\institute{Journal Club}

\ifpdf
  % Ensure reproducible output
  \pdfinfoomitdate=1
  \pdfsuppressptexinfo=-1
  \pdftrailerid{}
  \hypersetup{
    pdfcreator={},
    pdfproducer={}
  }
\fi

\begin{document}

{
  \begin{frame}{}
    \titlepage
  \end{frame}
}

\begin{frame}{Doppler effect}
  \begin{center}
    \begin{tikzpicture}
      \visible<-4>{
        \draw[blue,line width=1] (0, 0) circle (1.2);
        \draw[line width=1] (-0.45, -0.9) -- (0.45, -0.9);
        \draw[line width=1] (-0.45, 0.9) -- (0.45, 0.9);
        \draw[red,->,snakearrow] (0, -0.9) -- node[right=0.1] {$\hbar\omega$} (0, 0.9);

        \visible<2-3>{
          \draw[red,->,snakearrow,line width=1.5] (-4, 0) --
          node[above=0.1] {$\omega$} (-1.6, 0);
        }

        \visible<3->{
          \draw[->,blue,line width=2] (-1, -1.5) -- node[below=0.1] {$v$} (1, -1.5);
        }
        \visible<4->{
          \draw[red,->,snakearrow,line width=1.5] (-4, 0) --
          node[above=0.1] {$\omega+\dfrac{\omega v}{c}$} (-1.6, 0);
        }
      }

      \visible<5->{
        \draw[blue, line width=1, domain=-3:3,smooth,variable=\x]
        plot ({\x}, {(\x / 3)^2 + 0.9});
        \draw[blue, line width=1, domain=-3:3,smooth,variable=\x]
        plot ({\x}, {(\x / 3)^2 - 0.9});

        \draw[->, line width=1] (-4, -1) -- (4, -1) node[below] {$v$};
        \draw[->, line width=1] (0, -1) -- (0, 3) node[right] {$E$};

        \draw[<->, red, line width=0.7] (2.5, {(2.5 / 3)^2 - 0.9}) --
        node[right=0.1] {$\hbar\omega$}
        (2.5, {(2.5 / 3)^2 + 0.9});

        \begin{scope}[shift={(3.5, {0.8 - 0.9})},scale=0.27]
          \draw[blue,line width=1] (0, 0) circle (1.2);
          \draw[line width=1] (-0.55, -0.8) -- (0.55, -0.8);
          \fill[red] (0, -0.8) circle (0.3);
          \draw[line width=1] (-0.55, 0.8) -- (0.55, 0.8);
        \end{scope}

        \begin{scope}[shift={(3.5, {1.2 + 0.9})},scale=0.27]
          \draw[blue,line width=1] (0, 0) circle (1.2);
          \draw[line width=1] (-0.55, -0.8) -- (0.55, -0.8);
          \draw[line width=1] (-0.55, 0.8) -- (0.55, 0.8);
          \fill[red] (0, 0.8) circle (0.3);
        \end{scope}

      }
      \visible<6->{
        \draw[->,red,snakearrow,line width=1]
        (0.5, {(0.5 / 3)^2 - 0.9}) node[below=0.2] {$v$}
        -- (1.2, {(1.2 / 3)^2 + 0.9}) node[above=0.2] {$v+\dfrac{\hbar\omega}{mc}$};
      }
      \visible<7->{
        \node[below] at (0, -1.5) {$\omega+\dfrac{m}{2\hbar}\left(v+\dfrac{\hbar\omega}{mc}\right)^2-\dfrac{mv^2}{2\hbar}=\omega+\dfrac{\omega v}{c}+\dfrac{\hbar\omega^2}{2mc^2}$};
      }
    \end{tikzpicture}
  \end{center}
\end{frame}

\begin{frame}{Sideband}
  \begin{center}
    \begin{tikzpicture}
      \draw[blue, line width=1, domain=-2.5:2.5,smooth,variable=\x]
      plot ({\x}, {(\x / 2)^2 + 2.9});
      \draw[blue, line width=0.5] ({-sqrt(0.2) * 2}, 0.2 + 2.9) -- ({sqrt(0.2) * 2}, 0.2 + 2.9);
      \draw[blue, line width=0.5] ({-sqrt(0.6) * 2}, 0.6 + 2.9) -- ({sqrt(0.6) * 2}, 0.6 + 2.9);
      \draw[blue, line width=0.5] ({-sqrt(1.0) * 2}, 1.0 + 2.9) -- ({sqrt(1.0) * 2}, 1.0 + 2.9);
      \draw[blue, line width=0.5] ({-sqrt(1.4) * 2}, 1.4 + 2.9) -- ({sqrt(1.4) * 2}, 1.4 + 2.9);

      \draw[blue, line width=1, domain=-2.5:2.5,smooth,variable=\x]
      plot ({\x}, {(\x / 2)^2});
      \draw[blue, line width=0.5] ({-sqrt(0.2) * 2}, 0.2) -- ({sqrt(0.2) * 2}, 0.2);
      \draw[blue, line width=0.5] ({-sqrt(0.6) * 2}, 0.6) -- ({sqrt(0.6) * 2}, 0.6);
      \draw[blue, line width=0.5] ({-sqrt(1.0) * 2}, 1.0) -- ({sqrt(1.0) * 2}, 1.0);
      \draw[blue, line width=0.5] ({-sqrt(1.4) * 2}, 1.4) -- ({sqrt(1.4) * 2}, 1.4);

      \draw[<->, red, line width=0.7] (2.4, {(2.4 / 2)^2 + 2.9}) --
      node[right=0.1] {$\hbar\omega$}
      (2.4, {(2.4 / 2)^2});

      \begin{scope}[shift={(3.2, {0.9})},scale=0.27]
        \draw[blue,line width=1] (0, 0) circle (1.2);
        \draw[line width=1] (-0.55, -0.8) -- (0.55, -0.8);
        \fill[red] (0, -0.8) circle (0.3);
        \draw[line width=1] (-0.55, 0.8) -- (0.55, 0.8);
      \end{scope}

      \begin{scope}[shift={(3.2, {1.2 + 2.9})},scale=0.27]
        \draw[blue,line width=1] (0, 0) circle (1.2);
        \draw[line width=1] (-0.55, -0.8) -- (0.55, -0.8);
        \draw[line width=1] (-0.55, 0.8) -- (0.55, 0.8);
        \fill[red] (0, 0.8) circle (0.3);
      \end{scope}

      \draw[blue,->,line width=1] (-1.75, 1.8 + 2.9) -- (-1.75, 1.4 + 2.9);
      \draw[blue,->,line width=1] (-1.75, 0.6 + 2.9) -- (-1.75, 1.0 + 2.9);
      \node[blue] at (-1.75, 1.2 + 2.9) {$\hbar\omega_m$};

      \visible<2->{
        \draw[->,red,snakearrow,line width=1]
        (0.5, 0.6) -- (0.5, {0.8 + 2.9});

        \draw[blue, line width=1,samples=1000,domain=-0.2:0.2,smooth,variable=\e]
        plot ({-exp(-(\e / 0.03)^2) * 0.7 - 2.6}, {\e + 3.2 + 1.0});
        \draw[black, line width=1,samples=1000,domain=-0.2:0.2,smooth,variable=\e]
        plot ({-exp(-(\e / 0.03)^2) - 2.6}, {\e + 3.2 + 0.6});
        \draw[red, line width=1,samples=1000,domain=-0.2:0.2,smooth,variable=\e]
        plot ({-exp(-(\e / 0.03)^2) * 0.3 - 2.6}, {\e + 3.2 + 0.2});

        \node[below right] at (-2.4, -0.2) {Frequency:};
        \node[below right] at (-0.5, -0.2) {$\omega+n\omega_m$};
      }
      \visible<3->{
        \node[below right] at (-2.4, -0.8) {Strength:};
        \node[below right] at (-0.5, -0.8) {$\langle n|\ue^{\ui k\hat x}|n+\Delta n\rangle$};
      }
    \end{tikzpicture}
  \end{center}
\end{frame}

\begin{frame}{Lamb-Dicke parameter}
  \begin{center}
    \[\langle n|\ue^{\ui k\hat x}|n+\Delta n\rangle\]
    \visible<2->{
      \[\hat x=\sqrt{\frac{\hbar}{2m\omega}}\paren{a+a^\dagger}\]
    }
    \visible<3->{
      \[k\hat x=\eta\paren{a+a^\dagger}\]
      \[\eta\equiv k\sqrt{\frac{\hbar}{2m\omega}}=kz_0\]
    }
  \end{center}
\end{frame}

\end{document}
% &\langle n|\ue^{\ui k\hat x}|n+\Delta n\rangle\\
% =&\langle n|\exp\paren{\ui\eta\paren{a+a^\dagger}}|n+\Delta n\rangle
