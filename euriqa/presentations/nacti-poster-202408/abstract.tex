%%% This is the template for abstract submission to NACTI 24
%%% Adapted from NACTI 2017 template, with permission.
%%% Based on the ECTI 16 template, with permission.
%%% 
%%% IMPORTANT INFORMATION
%%% Please read the instructions for each field carefully.
%%% 
%%% Please adjust the length of your abstract to fit the frame
%%% displayed when it is compiled to pdf.
%%% 

\documentclass{nacti2024}

\begin{document}

%% Enter name and e-mail address of the corresponding author here:
\CorrespondingAuthorGivenName{Yichao}
\CorrespondingAuthorSurname{Yu}
\CorrespondingAuthorEmail{yichao.yu@duke.edu}

%%% Select type of presentation: "poster", or "invited"
%%% If you applied to give a hot topics talk, please use "poster" type nonetheless
%%% NB invited talks have bigger boxes

\PresentationType{poster}
% \PresentationType{invited}

%% Enter title of contribution:

\Title{Mid-circuit partial measurement on Yb171 using the OMG architecture}

%% Enter the list of the authors and their affiliations. Please underline the name of the corresponding author.

\AuthorsList{\underline{Yichao Yu}$^1$, Keqin Yan$^1$, Debopriyo Biswas$^1$,
  Vivian Zhang$^1$, Bahaa Harraz$^1$, Marko Cetina$^1$, Crystal Noel$^1$,
  Alexander Kozhanov$^1$, Christopher R Monroe$^1$}

\Affiliations{
  $^1$Duke Quantum Center, Durham, NC 27701\\
}

\Abstract{
  Error correction is an important ingredient for achieving scalable and fault-tolerant
  quantum computing in the long term. Despite the recent experimental advances
  in error correction, however, mid-circuit partial measurement and reset of the
  physical qubit remains one of the main technical challenges on atomic platforms.

  The existing approaches for achieving mid-circuit partial measurement and reset on
  trapped ion platforms mainly relies on the shuttling of the ions to avoid
  destroying the quantum information during the dissipative process, which are slow
  and could take up a significant portional of the runtime and causes the ion to heat up
  during shuttling. Here we present our recent progress on the implementation
  of mid-circuit measurement methods based on the optical-metastable-ground (OMG)
  architecture using the metastable states in Yb-171 ions.
}

\createabstract
\end{document}
