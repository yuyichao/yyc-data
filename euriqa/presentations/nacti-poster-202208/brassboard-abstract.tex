\documentclass[superscriptaddress]{revtex4-2}

\begin{document}

%% -- Please don't edit above this line

%% -- Title
\title{A next-generation trapped ion quantum computing system}

%% -- Authors and affiliations
\author{\underline{Yichao Yu}} % Presenter
\affiliation{DQC/Duke ECE}
\author{Liudmila Zhukas}
\affiliation{DQC/Duke ECE}
\author{Lei Feng}
\affiliation{JQI/QuICS/UMD Physics}
\affiliation{DQC/Duke ECE}
\author{Marko Cetina}
\affiliation{JQI/QuICS/UMD Physics}
\affiliation{DQC/Duke ECE}
\author{Crystal Noel}
\affiliation{JQI/QuICS/UMD Physics}
\affiliation{DQC/Duke ECE}
\author{Debopriyo Biswas}
\affiliation{JQI/QuICS/UMD Physics}
\affiliation{DQC/Duke ECE}
\author{Andrew Risinger}
\affiliation{JQI/QuICS/UMD Physics}
\author{Alexander Kozhanov}
\affiliation{DQC/Duke ECE}
\author{Christopher Monroe}
\affiliation{JQI/QuICS/UMD Physics}
\affiliation{DQC/Duke ECE}
\affiliation{IonQ}

%% -- Abstract
\begin{abstract}
  The first generation of a universal trapped ion integrated quantum processor, constructed in a collaboration between universities and industrial partners, was used to perform quantum algorithms with high-fidelity on 12 qubits, and high-fidelity quantum gates with up to 23 qubits. We present progress on the second-generation system, which has several design improvements, such as a capacity of 32 qubits, parallel addressing capability using an RF-System-On-Chip, a next-generation micro-fabricated surface ion trap from Sandia National Laboratories, and the integration with the upgraded Raman and CW laser systems built by L3Harris.
\end{abstract}

\maketitle

\end{document}
